\part{Business Understanding}

    The pervasive issue of obesity is not just about weight—it has been linked to severe health risks like diabetes,
    high blood pressure, mental health, and cardiac problems. Through data mining techniques, we aim to pinpoint
    practical approaches to counter and manage obesity, thus enhancing overall public health.
    \\
    \\
    Combatting obesity is a pivotal step towards support public health. By relying on proven methods or patters,
    which we aim to find, we can anticipate issues, foster healthier societies, and relieve some stress from our
    medical infrastructures. Unraveling the diverse factors behind obesity, from mental states to daily habits, is
    central to this endeavor. With such knowledge, authorities can optimize resource allocation and shape strategies
    that resonate with specific locales.
    \\
    \\
    The prevalence of obesity is inconsistent across nations, underscoring the need for customized approaches. A
    thorough investigation of factors in psychological triggers, eating habits, and socioeconomic factors can present
    a well-rounded perspective on the issue. When there might be gaps in data for certain countries or specific years,
    a common year with comprehensive data can be selected. For any data voids, techniques that leverage average values
    from other years might be a solution.
    \\
    \\
    Our mission is to bring down obesity figures to at most 20\% in countries that presently overshoot this number.
    Grasping the elements that mold these statistics is fundamental to tweaking our strategies. A look into the success
    stories of other nations can provide countries with the knowledge to uplift their health standards.
    \\
    \\
    Employing data mining, we aim to uncover the dominant factors driving obesity. Detailed statistical methods aid
    in this exploration. Even though we have rich datasets on mental health and dietary habits, they might not be
    exhaustive. This gap makes a strong case for using a blend of analytical tools. Furthermore, considering a broader
    range of indicators can further fine-tune our strategy formulation.
    \\
    \\
    In this statistical analysis, informed by the expertise of a general practitioner who participates actively in
    nutrition, we postulate that obesity stems not solely from consumption habits but also from the mental well-being
    of individuals. These factors, which vary by country, influence obesity rates. Additionally, it’s worth noting that
    better obesity rates in some nations might be misleading, as they could be attributed to food scarcity and prevalent
    hunger resulting from socio-economic challenges faced by those nations.


    \section{Project Plan}
        \begin{table}[ht]
            \centering
            \begin{tabularx}{\textwidth}{lcc}
                \toprule
                \textbf{Phase Name}   & \textbf{Time Allocated (\%)} & \textbf{Description of Risks}                             \\
                \midrule
                Data Understanding    & 10\%                         & Incomplete or inconsistent data, lack of domain expertise \\
                Data Preparation      & 25\%                         & Data quality issues, missing values, imbalanced data      \\
                Data Transformation   & 10\%                         & Incorrect transformations, loss of key information        \\
                Data Mining Methods   & 10\%                         & Selection of inappropriate method                         \\
                Data Mining Algorithm & 10\%                         & Algorithm doesn't converge, overfitting issues            \\
                Data Mining           & 20\%                         & Model doesn't generalize well, low accuracy               \\
                Interpretation        & 15\%                         & Misinterpretation of results, not actionable insights     \\
                \bottomrule
            \end{tabularx}
            \caption{Project Plan based on CRISP-DM methodology}
            \label{table:projectplan}
        \end{table}

        \subsection{Day-to-Day Timeline}

            Given a 3-week (about 21-day) cycle for each iteration:

            \begin{description}
                \item[Days 1:] Data Understanding
                \item[Days 2-5:] Data Preparation
                \item[Days 6-8:] Data Transformation
                \item[Days 9-11:] Data Mining Methods
                \item[Days 12-14:] Data Mining Algorithm
                \item[Days 15-18:] Data Mining
                \item[Days 19-21:] Interpretation
            \end{description}

            This cycle will repeat for each iteration of the project, ensuring a systematic approach to data mining.
